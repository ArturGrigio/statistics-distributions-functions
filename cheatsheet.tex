% Preamble
% ---

\documentclass[8pt]{extarticle}

% Packages
% ---

\usepackage{amsmath}
\usepackage{amssymb}
\usepackage[margin=0.5in]{geometry}
\usepackage{empheq}
\usepackage{pgfplots}

\begin{document}

\section{Combinatorics}

\begin{center}
    \begin{tabular}{ | l | l | l | }
        \hline
        & With Replacement & Without Replacement \\ \hline

        Permutations (Ordered)   & $P_{r}(n,k)=n^k$              & $P(n,k)=\frac{n!}{(n-k)!}$ \\ \hline
        Combinations (Unordered) & $C_{r}(n,k)=\binom{n+k-1}{k}=\frac{(n+k-1)!}{k!(n-1)!}$ & $C(n,k)=\binom{n}{k}=\frac{n!}{k!(n-k)!}$ \\ \hline
    \end{tabular}
\end{center}

\section{Distributions}

\begin{center}
    \begin{tabular}{ | l || l || l | l | l | l || l | l | l | }
    \hline
    Distribution & Moments & Bernoulli & Binomial & Geometric & Poisson & Exponential & Uniform & Normal \\ \hline

    PDF &
        & $p^x(1-p)^{1-x}$
        & $\binom{n}{x}p^x(1-p)^{n-x}$
        & $(1-p)^{x-1}p$
        & $\mathrm{e}^{-\lambda}\frac{\lambda^x}{x!}$
        & $\frac{1}{\lambda}\mathrm{e}^{\frac{-x}{\lambda}}$
        & $\frac{1}{b-a}$
        & $\frac{1}{\sqrt{2\pi\sigma^2}}\mathrm{e}^{\frac{-(x-\mu)^2}{2\sigma^2}}$ \\ \hline

    $\mathrm{E[X]}$ & $\bar{X} = \frac{1}{n}\displaystyle\sum_{i=1}^{n} X_i$
                    & $p$
                    & $np$
                    & $\frac{1}{p}$
                    & $\lambda$
                    & $\lambda$
                    & $\frac{a+b}{2}$
                    & $\mu$ \\ \hline

    $\mathrm{Var[X]}$ & $E[X^2] = \frac{1}{n}\displaystyle\sum_{i=1}^{n} (X_i - \bar{X})^2$
                      & $p(1-p)$
                      & $np(1-p)$
                      & $\frac{1-p}{p^2}$
                      & $\lambda$
                      & $\lambda^2$
                      & $\frac{(b-a)^2}{12}$
                      & $\sigma^2$ \\ \hline
    \end{tabular}
\end{center}

\pgfmathdeclarefunction{gauss}{2}{%
  \pgfmathparse{1/(#2*sqrt(2*pi))*exp(-((x-#1)^2)/(2*#2^2))}%
}

\section{Percentiles; $Q \sim Normal(\mu_Q=80, \sigma^2_Q=9)$; percentile = 97.5}
\fbox{
     \addtolength{\linewidth}{-2\fboxsep}%
      \addtolength{\linewidth}{-2\fboxrule}%
       \begin{minipage}{\linewidth}
            \begin{equation}
                P(Q \leq q_0.975) = 0.975  \\
            \end{equation}
            \begin{equation}
                \begin{tikzpicture}
                    \begin{axis}[every axis plot post/.append style={
                        mark=none,domain=65:90,samples=50,smooth},
                        axis x line*=bottom,
                        axis y line*=left,
                        enlargelimits=upper]
                        \addplot {gauss(80,3)};
                    \end{axis}
                \end{tikzpicture}
            \end{equation}
            \begin{equation}
                P (Z \leq \frac{q_0.975 - 80}{3}) = 0.975 \\
            \end{equation}
            \begin{equation}
                \frac{q_0.975 - 80}{3} = 1.96 \\
            \end{equation}
            \begin{equation}
                q_0.975 \approx 85.88\\
            \end{equation}
            \text{So $x_Q$ does not fall within percentile if $x_Q < q_0.975$}
        \end{minipage}
}

\section{Proportion within range; $X \sim Normal(\mu_Q=1000, \sigma^2_Q=625); n=10$}
\fbox{
     \addtolength{\linewidth}{-2\fboxsep}%
      \addtolength{\linewidth}{-2\fboxrule}%
       \begin{minipage}{\linewidth}
            \text{What proportion of bricks weigh between 950 to 1050 grams?}
            \begin{equation}
                {P}(950 \leq X \leq 1050) = P(\frac{950 - 1000}{25} \leq Z \leq \frac{1050 - 1000}{25}) = P(-2 \leq Z \leq 2) = P(Z \leq 2) - P(Z \geq -2) = .97725 - .0228 \approx .95
            \end{equation}
            \text{what is the probability that all of them pass
the weight requirement?}
            \begin{equation}
                Y_i = \begin{cases}
                    1 & 950 \leq X_i \leq 1050 \\
                    0 & \text{otherwise}
                \end{cases}
            \end{equation}
            \begin{equation}
                P(Y_i = 1) = P(950 \leq X_i \leq 1050) = .95
            \end{equation}
            \begin{equation}
                \text{let } Y_i = Y_1 + ... + Y_10 = \text{bricks that meet requirement}
            \end{equation}
            \begin{equation}
                P(Y = 10) = (.95)^{10}
            \end{equation}
            \text{What is the probability that exactly 9 out of 10 meet the requirement?}
            \begin{equation}
                P(Y = 9) = {10 \choose 9}\times(.95)^9\times(1-.95)^{10-9}
            \end{equation}
            \text{What is the probability that at least one fails the requirement?}
            \begin{equation}
                P(Y < 10) = 1 - P(Y = 10) = 1 - (.95)^{10}
            \end{equation}
            \text{ Suppose it is known that there was exactly one failure. What is the probability that the failure was due to brick number 1?}
            \begin{equation}
                n = 10 \iff answer = \frac{1}{10}
            \end{equation}
            \text{if testing was done sequentially, What is the probability that the first failure is incurred by the fourth brick
            tested?}
            \begin{equation}
                P(Y_1=1, Y_2=1, Y_3=1, Y_4=0) = P(Y_1=1)\times...\times P(Y_4=0) = (.95)^3\times(.05)
            \end{equation}
            \text{What is the probability that the second failure occurs on the 10-th brick. Hint: exactly one failure in first 9 bricks and the 10-th brick is failure.}
            \begin{equation}
                P(\{\text{one failure in 9 bricks}\} \cap \{Y_{10} = 0\}) = ({9 \choose 1} \times (.95)^8 \times .05) \times .05 = 9 \times .95^8 \times .05^2
            \end{equation}
        \end{minipage}
}

\section{Lifetimes; $X \sim Exponential(\lambda_X=1000), Y \sim Exponential(\lambda_Y=1200)$ }
\fbox{
    \addtolength{\linewidth}{-2\fboxsep}%
    \addtolength{\linewidth}{-2\fboxrule}%
    \begin{minipage}{\linewidth}
        \text{Given industry standard is $\lambda=400$ derive CDF}
        \begin{equation}
        F_X(x) = \int_{0}^{x} f_X(x) dx
        = \int_{0}^{x} \frac{e^{\frac{-t}{\lambda_X}}}{\lambda_X} dt
        = -e^{\frac{-t}{\lambda_X}}\Big|_{0}^{x}
        \end{equation}
        \text{What is the proportion of devices produced by Company-X that meets that minimum standard?}
        \begin{equation}
            P(X \geq 400) = 1 - F_X(X = 400) = e^{\frac{-400}{1000}} \approx .67
        \end{equation}
        \text{What is the proportion of devices produced by Company-Y that meets that minimum standard?}
        \begin{equation}
            P(Y \geq 400) = 1 - F_Y(X = 400) = e^{\frac{-400}{1200}} \approx .717
        \end{equation}
        \text{What is the proportion of devices manufactured by Company-X that gives lifetimes between 400 and 1600 hours?}
        \begin{equation}
            P(400 \leq X \leq 1600) = F_X(X=1600) - F_X(X=400) = e^{\frac{-400}{1000}} - e^{\frac{-1600}{1000}} \approx .468
        \end{equation}
        \text{Given Company-X makes 0.45/Company-Y makes 0.55 of devices. A device tested passed the minimum standard. Guess which company made that device. Justify.}
        \begin{equation}
            T = \text{device that passed}
        \end{equation}
        \begin{equation}
            C_X = \text{device produced by company X}\\
        \end{equation}
        \begin{equation}
            C_Y = \text{device produced by company Y}\\
        \end{equation}
        \begin{equation}
            P(C_X) = .45
        \end{equation}
        \begin{equation}
            P(C_Y) = .55
        \end{equation}
        \begin{equation}
            P(T|C_X) = P(X \geq 400) = .67
        \end{equation}
        \begin{equation}
            P(T|C_Y) = P(Y \geq 400) = .717
        \end{equation}
        \begin{equation}
            P(T) = P(T|C_X) \times p(C_X) + P(T|C_Y) \times P(C_Y) = (.67)(.45) + (.717)(.55) = .696
        \end{equation}
        \begin{equation}
            P(C_X|T) = \frac{P(T|C_X)P(C_X)}{P(T)} = \frac{(.67)(.45)}{.696} = .433
        \end{equation}
        \begin{equation}
            P(C_Y|T) = 1 - P(C_X|T) = 1 - .433 = .567 \therefore \text{the device was probably made by Company Y}
        \end{equation}
    \end{minipage}
}

\section{Hypothesis Testing}

\begin{center}
    \begin{tabular}{ | l | l | l | l | l | l | }
        \hline
        $\mathrm{H}_0$ & $\mathrm{H}_A$ & Conditions & Test Statistic & Rejection Region & Degrees of freedom \\ \hline

        % ROW 1
        % ---
        & $\mu_1<\mu_2$
        &
        &
        & $\mathrm{Z}<-\mathrm{z}_\alpha$
        & \\
        % ---

        % ROW 2
        % ---
        $\mu_1=\mu_2$
        & $\mu_1>\mu_2$
        & Known $\sigma$
        & $\mathrm{Z}=\frac{(\bar{X}_1-\bar{X}_2)-(\mu_1-\mu_2)}{\sqrt{\frac{\sigma_1^{2}}{n_1}+\frac{\sigma_2^{2}}{n_2}}}$
        & $\mathrm{Z}>\mathrm{z}_\alpha$
        & \\
        % ---

        % ROW 3
        % ---
        & $\mu_1\ne\mu_2$
        &
        &
        & $\mathrm{|Z|}>\mathrm{|z_{\alpha/2}|}$
        & \\ \hline\hline
        % ---

        % ROW 4
        % ---
        & $\mu_1<\mu_2$
        & Unknown but equal $\sigma$
        &
        & $\mathrm{t}<-\mathrm{t}_{df,\alpha}$
        & \\

        % ROW 5
        % ---
        $\mu_1=\mu_2$
        & $\mu_1>\mu_2$
        & $s^2 = \frac{1}{n-1}\displaystyle\sum_{i=1}^{n} (x_i - \bar{x})^2$
        & $\mathrm{t}=\frac{(\bar{X}_1-\bar{X}_2)-(\mu_1-\mu_2)}{\sqrt{{s_p}^2(\frac{1}{n_1}+\frac{1}{n_2})}}$
        & $\mathrm{t}>\mathrm{t}_{df,\alpha}$
        & $\mathrm{df}={n_1+n_2-2}$ \\
        % ---

        % ROW 6
        % --
        & $\mu_1\ne\mu_2$
        & $s^2_p = \frac{(n_1-1)s^2_1 + (n_2-1)s^2_2}{n_1 + n_2 - 2}$
        &
        & $\mathrm{|t|}>\mathrm{|t}_{df,\alpha/2}|$
        & \\ \hline\hline
        % --

        % ROW 7
        % ---
        & $\pi_1<\pi_2$
        &
        &
        & $\mathrm{Z}<-\mathrm{z}_\alpha$
        & \\
        % ---

        % ROW 8
        % ---
        $\pi_1=\pi_2$
        & $\pi_1>\pi_2$
        & Z-test for proportions
        & $\mathrm{Z}=\frac{(\hat{\pi}_1-\hat{\pi}_2)-(\pi_1-\pi_2)}
        {\sqrt{\frac{\hat{\pi}_{1}(1-\hat{\pi}_1)}{n_1}+\frac{\hat{\pi}_{2}(1-\hat{\pi}_2)}{n_2}}}$
        & $\mathrm{Z}>\mathrm{z}_\alpha$
        & \\
        % ---

        % ROW 9
        % ---
        & $\pi_1\ne\pi_2$
        &
        &
        & $\mathrm{|Z|}>\mathrm{|z_{\alpha/2}|}$
        & \\ \hline\hline

        % ROW 10
        % ---
        &
        &
        &
        &
        & \\
        % ---

        % ROW 11
        % ---
        $\beta_1=0$
        & $\beta_1\ne0$
        & Test for linear relationship
        & $\mathrm{t}=\frac{\hat{\beta}_1-\beta_1}{\sqrt{\frac{s^2}{S_{xx}}}}$
        & $\mathrm{t}>\mathrm{|t}_{df,\alpha/2}|$
        & $\mathrm{df}={n-2}$ \\
        % ---

        % ROW 12
        % ---
        &
        &
        &
        &
        & \\ \hline
        % ---

    \end{tabular}
\end{center}

\section{Linear Regression Formulas}

\fbox{
     \addtolength{\linewidth}{-2\fboxsep}%
      \addtolength{\linewidth}{-2\fboxrule}%
       \begin{minipage}{\linewidth}
             \begin{equation}
                    \hat{\beta}_0 = \bar{y} - \hat{\beta}_1 \bar{x} \\
                      \end{equation}
                \begin{equation}
                    \hat{\beta}_1 = \frac{S_{xy}}{S_{xx}} \\
                      \end{equation}
                \begin{equation}
                    S_{xx}=\displaystyle\sum_{i=1}^{n} (x_i - \bar{x})^2\\
                      \end{equation}
                \begin{equation}
                    S_{xy}=\displaystyle\sum_{i=1}^{n} (x_i - \bar{x})(y_i - \bar{y})\\
                      \end{equation}
                \begin{equation}
                    SS_{RESID}=\displaystyle\sum_{i=1}^{n} (y_i - \hat{y})\\
                      \end{equation}
                \begin{equation}
                    Regression Variance: s^2 = \frac{SS_{RESID}}{n-2} \\
                      \end{equation}
                \begin{equation}
                    Normal Equations: \begin{Bmatrix}
                                        \displaystyle\sum_{i=1}^{n} (y_i - \beta_0 - \beta_1 x_i) = 0 \\
                                        \displaystyle\sum_{i=1}^{n} x_i(y_i - \beta_0 - \beta_1 x_i) = 0 \\
                                        \end{Bmatrix}
                      \end{equation}

                       \end{minipage}
                   }
\end{document}
